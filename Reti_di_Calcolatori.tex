\documentclass{article}

\usepackage{cancel}
\usepackage{amsmath}
\usepackage[includehead,nomarginpar]{geometry}
\usepackage{graphicx}
\usepackage{amsfonts} 
\usepackage{verbatim}
\usepackage{mathrsfs}  
\usepackage{lmodern}
\usepackage{braket}
\usepackage{bookmark}
\usepackage{fancyhdr}
\usepackage{romanbarpagenumber}
%\usepackage{minted}
%\usepackage{subfig}
\usepackage[italian]{babel}
%\usepackage{float}
%\usepackage{wrapfig}
%\usepackage[export]{adjustbox}
\allowdisplaybreaks

\setlength{\headheight}{12.0pt}
\addtolength{\topmargin}{-12.0pt}
\graphicspath{ {./Immagini/} }

\hypersetup{
    colorlinks=true,
    linkcolor=black,
}

\newsavebox{\tempbox} %{\raisebox{\dimexpr.5\ht\tempbox-.5\height\relax}}


\makeatother

\numberwithin{equation}{subsection}
\newcommand{\tageq}{\tag{\stepcounter{equation}\theequation}}
\AtBeginDocument{%
  \renewcommand{\figurename}{Fig.}
}
\fancypagestyle{link}{\fancyhf{}\renewcommand{\headrulewidth}{0pt}\fancyfoot[C]{Sorgente del file LaTeX disponibile al seguente link: \url{https://github.com/00Darxk/Reti-di-Calcolatori/}}}

\begin{document}

\title{%
    \textbf{Reti di Calcolatori}  \\ 
    \large Appunti delle Lezioni di Reti di Calcolatori \\
    \textit{Anno Accademico: 2024/25}}
\author{\textit{Giacomo Sturm}}
\date{\textit{Dipartimento di Ingegneria Civile, Informatica e delle Tecnologie Aeronautiche \\
Università degli Studi ``Roma Tre"}}

\maketitle
\thispagestyle{link}

\clearpage


\pagestyle{fancy}
\fancyhead{}\fancyfoot{}
\fancyhead[C]{\textit{Reti di Calcolatori - Università degli Studi ``Roma Tre"}}
\fancyfoot[C]{\thepage}
\pagenumbering{Roman}

\tableofcontents

\clearpage
\pagenumbering{arabic}

\section{Introduzione}

Una qualsiasi interconnessione di calcolatori può rappresentare una rete di calcolatori, ma in base alla distanza reciproca tra questi componenti 
si tratta di reti differenti. Convenzionalmente si considerano reti di calcolatori, sistemi di calcolatori interconnessi ad una distanza 
superiore ai 50 cm. Una distanza minore, fino ai 5 cm, generalmente interessa componenti dello stesso computer, sulla stessa scheda madre, connesse tra di loro; mentre una 
distanza inferiore ai 5 cm rappresenta componenti sullo stesso chip. Inoltre le reti considerate possono essere ulteriormente divise in base 
alla distanza dei loro elementi:
\begin{itemize}
  \item Se hanno una distanza minore di 5 km, si tratta di risorse connesse sulla stessa rete o edificio, o su edifici vicini. Questo tipo di rete si chiama Local Area Network (LAN);
  \item Se hanno una distanza superiore ai 5 km, si tratta di risorse connesse su una vasta area geografica. Questo tipo di rete si chiama Wide Area Network (WAN).
\end{itemize}
Tra questi due livelli possono essere presenti anche tecnologie molto diverse tra di loro, queste tecnologie vengono identificate da acronimi da cui 
è possibile ricavare lo scopo della tecnologia, senza tuttavia conoscere il suo funzionamento. 

Una connessione tra componenti di una rete coinvolge sempre uno scambio di informazioni, tramite uno scambio di messaggi in serie. Gli elementi della 
rete effettuano degli accessi ad essa apparentemente in parallelo e simultanei, per poter comunicare tra di loro. Mentre su componenti sulla stessa macchina o sullo 
stesso chip avvengono tramite accessi ad una memoria condivisa. 

Le connessioni componenti di una rete avvengono su uno strato fisico, quindi attraverso diversi mezzi trasmissivi, i quali non verranno analizzati approfonditamente a 
questo livello di astrazione. Tra i più comuni mezzi trasmissivi abbiamo cavi in fibra ottica, o in rame, ed onde radio. 

\subsection{Commutazione}

All'interno di una rete si possono utilizzare due tipi diversi di commutazione, di circuito o di pacchetto. Il termine commutazione risale alla telefonia, quando 
diverse aree telefoniche dovevano essere collegate tra di loro, un operatore ad un centralino doveva manualmente collegare con un cavo le due aree interessate. 
Quest'operazione di connettere elementi tramite nodi intermedi in una rete rappresenta un elemento comune di tutte le reti. Collegare singolarmente tutti gli 
$n$ elementi di una rete comporterebbe un numero totale di $n(n-2)/2$ connessioni. All'aumento dei componenti in una rete il numero $n$ di connessioni da costruire sulla 
rete aumenta quadraticamente $O(n^2)$. Per cui è molto più conveniente mantenere nodi intermedi comuni a molti percorsi tra componenti della rete, utilizzando 
apparecchiature intermedie per realizzare circuiti. In questo modo però non è possibile soddisfare tutte le possibili coppie contemporaneamente, ma si elimina la 
crescita quadratica del sistema. 
La commutazione di circuito consiste nella creazione del percorso fisico che connette due elementi della rete. 


%% todo aggiungere lezioni 25, 27 sett; 2, 4, 9 ott

\section{Standard IEEE 802}

%% add

\subsection{802.11: WiFi}

%% add

\subsection{802.3: Ethernet}

%% add

Un pacchetto ethernet deve comunque avere 512 bit da trasmettere, altrimenti la parte residua del campo dati deve essere riempita di bit 
di riempimento chiamato padding, inoltre il campo dati può contenere al massimo 1500 Byte. 

\subsection{802.1D: Bridge-Switch}

La connessione instaurata tramite ethernet è bidirezionale simultanea solamente su due calcolatori, ma su una stessa connessione può inviare i dati un solo 
calcolatore, quindi sono necessarie altre componenti. Un bridge è una componente che consente di connettere tra di loro più di un computer 
tramite ethernet, comportandosi come se fosse un calcolatore intermedio ai calcolatori della rete, connesso a ciascuno di questi tramite una connessione 
ethernet. 

Inoltre connettendo tra di loro diversi bridge è possibile creare una struttura più articolata, creando una struttura simile ad un albero. La 
parte wired o cablata delle connessioni LAN vengono instaurate in questo modo. 

%% todo img rete bridge


I bridge svolgono una prima funzione di rendere possibili topologie articolate, effettuando un'operazione di ``filtering'', per separare tra di loro porzioni di rete che 
non devono dialogare tra di loro in modo diretto

%% todo img filtering

I bridge sono delle macchine ``store \& forward'', ovvero quando ricevono un pacchetto, prima di essere inviato su altre porte, viene 
memorizzato e trasmesso su altre porte, analogamente come se fosse un calcolatore, in caso le altre porte siano impegnate a trasmettere altri 
pacchetti, quindi in caso di traffico. Si può quindi immaginare una coda di pacchetti sulle porte del bridge per essere trasmesse. 

Il bridge sono delle tecnologie di livello 2, ed utilizzano algoritmi di instradamento per inviarli ad un MAC address specifico, ma questo tipo di 
algoritmo viene effettuato a livello 3. Questo non sorge problemi, poiché quest'operazione di instradamento è interna alla LAN, e non coinvolge alcun'altra componente 
della rete. I bridge devono essere conformi allo standard IEEE 802.1D. Gli standard comprendenti il carattere ``D'', sono di grande importanza. 
I sistemi connessi a reti LAN ignorano i bridge, si dicono quindi trasparenti, poiché i calcolatori connessi alla rete non conoscono la loro posizione all'interno della 
rete. 

%% todo img connessione lan dallo standard 

Un calcolatore per inviare un messaggio ad un altro calcolatore su una rete LAN, invia il suo pacchetto ad un bridge attraverso il sul MAC address. Il bridge quindi 
utilizza in principio un diverso MAC address, per spedire questo pacchetto al computer di destinazione tramite il suo MAC address. Tra questi due MAC è presenta un 
componente di relay, per trasmettere il pacchetto tra porte diverse del bridge. 


Le porte di un bridge possono avere lo stesso MAC o MAC differenti. Poiché il pacchetto è specifico al MAC del bridge, deve ricostruire il pacchetto scartando i campi 
specifici al MAC address del bridge. Inoltre poiché i pacchetti non sono tutti conformi allo standard IEEE 802.3, deve ricostruire anche il campo LLC. I MAC address 
dei computer nella rete sono realizzati in modo da poter essere connessi a ciascun tipo di MAC. 

%%\subsubsection{Learning}

Si vuole che il modello di rete sia ``plug \& play'', ovvero non deve essere dipendente da un intervento umano. 
I bridge costruiscono la loro tabella di instradamento per identificare dove sono presenti i diversi MAC address, autonomamente attraverso un meccanismo di ``learning'', 
salvando questa tabella nel ``filtering database''. Ogni porta del bridge rappresenta una linea ethernet diversa, identificando un loro dominio di collisione, a cui 
possono essere connessi diversi calcolatori. 

Si considera una rete dove ogni componente connesso è spento, ed una tabella vuota. Appena si accende un calcolatore ed invia un pacchetto da un dominio di collisione, 
allora il bridge capisce a quale porta corrisponde il MAC address del mittente. Ma ancora non conosce dove si trova il destinatario, quindi lo invia su tutte le sue 
porte disponibili, su tutta la rete. Invece se conosce la porta dov'è presente il destinatario lo invia solamente su quella porta. 

%% todo img tabella di filtering

Il learning permette di costruire autonomamente il filtering database di un bridge, questo meccanismo tuttavia non funziona quando al rete presenta una topologia diversa 
dalla topologia ad albero. Per esempio se è presente un ciclo all'interno della rete, il bridge si vede arrivare un pacchetto dallo stesso MAC address su porte diverse. 
Ma un albero è una topologia contenente solo ``Single Points of Failures'' (SPoF) e quindi fortemente sconsigliata, poiché un singolo malfunzionamento causerebbe la 
perdita di funzionalità dell'intera rete. Per cui data una topologia a grafo, un bridge è in grado di calcolare autonomamente un albero 
ricoprente della rete, ad ogni cambio di topologia della stessa. I bridge inoltre vengono collegati tra di loro per più di una connessione per evitare altri SPoF, ed 
evitare che a un singolo guasto la rete venga tagliata in due. 

Tramite un meccanismo progressivo i bridge individuano la loro posizione nella struttura dell'albero ricoprente e sono in grado di staccare alcune porte e rimanere 
collegati sull'intera rete; quando questi bridge rilevano un guasto su una di queste connessioni, riattivano una delle porte disattivate per mantenere in funzione la 
rete, ottenendo una significativa resistenza ai gusti. 
Questo tipo di algoritmo di spanning tree verrà trattato in corsi più avanzati di reti di calcolatori. 

Le prestazioni di un bridge influenzano le prestazioni dell'intera rete locale, vengono identificate da una serie di parametri. Il numero massimo di pacchetti al 
secondo processabili dal bridge, rappresentano un collo di bottiglia per i pacchetti che possono essere presenti sulla rete in ogni singolo momento, se vengono 
inviati un numero superiore ci pacchetti, alcuni pacchetti verranno scartati. Un altro parametro caratteristico è il tempo medio di latenza, ovvero il tempo in cui il 
bridge prende le sue decisioni ed invia il pacchetto alla porta giusta. 
Per cui è preferibile avere bridge full speed, ovvero con una velocità pari al massimo teorico. Più corti sono i pacchetti maggiore è il numero di decisioni 
effettuate nell'unità di tempo. Nello standard IEEE 802.3 a 10 Mb/s, un bridge si definisce full speed, se è in grado di processare 41880 pacchetti al secondo, 
per ogni porta. Questi esperimenti di verifica a parità di frequenza devono essere effettuati utilizzando pacchetti di lunghezza minima, così ad ogni porta è presente 
la massima frequenza di funzionamento del bridge. 
Il pacchetto più piccolo che può essere inviato è da 512 bit, per cui il numero massimo di pacchetti al secondo ad una 
velocità di 10 Mb/s è di circa 19500 pacchetti, ma il pacchetto comprende anche il preambolo ed lo SFD, per cui vanno aggiunti altri 64 bit, 
numero di pacchetti al secondo scende quindi a 17300. 
Tuttavia tra un pacchetto ed il successivo in ethernet è presente l'``interpacket gap'' di 96 bit. 
Per ciascuna tipologia di porta del bridge si effettua questa analisi e si verifica nel caso peggiore quanti pacchetti è in grado di gestire. 


Il bridge è un calcolatore, con una CPU, RAM ed interfacce per le diverse LAN, in ROM le funzionalità dello standard %%
Per bridge più potenti, le  porte vengono raelizzate tramite schede ASIC, per risolvere il problema dell'instradamento localmente. Le porte vengono realizzati tramite 
diversi slot che possono essere inseriti o rimossi in base al tipo di porta necessaria. Inoltre sono necessari massicci impianti di raffreddamento per riuscire a 
mantenere la temperatura del data center. A differenza di un server che può essere rallentano in caso di traffico allentato e quindi diminuire la temperature, le apparecchiature di bridge non possono 
spegnersi, e quindi comportano una temperatura costantemente elevata. Bridge di fascia alta sono in grado di effettuare bilanci sulle prese di corrente. 

Logicamente sono presenti almeno due porte una ``MAC relay entity'', per trasmettere i pacchetti tra le varie porte, ed un'entità di livello superiore, per la gestione 
del bridge, degli algoritmi e dei protocolli. Queste entità di alto livello comunicano con altri bridge attraverso pacchetti per realizzare lo spanning tree. 

Le porte del bridge possono essere abilitate o disattivate dall'amministratore di rete. Una porta attiva può essere in stato di ``forwarding'' o di ``blocking'', se sono 
bloccate lo spanning tree lo ha bloccate. Ogni porta ha un indirizzo MAC univoco, e sono numerate progressivamente a partire da uno. Convenzionalmente l'indirizzo MAc 
del bridge corrisponde all'indirizzo MAC della porta numero uno. 

La tabella di instradamento contiene ``entries'' (righe) statiche o dinamiche. Le righe statiche vengono inserite dall'amministratore a causa di esigenze di sicurezza 
importanti, altrimenti la posizione del MAC address vengono mantenuti per un tempo finito, configurabile, e di default di 5 minuti. Infatti è possibile che il calcolatore 
venga spostato spazialmente attraverso la rete, è quindi possibile si colleghi ad una porta differente. 

%% add ??

Lo sviluppo di ethernet ha portato alla creazione di meccanismi di controllo di flusso, soprattutto per gli switch. Una richiesta attraverso il bridge può essere di 
pochi byte verso un server, ma può provocare un trasferimento notevole di dati verso il client, quindi attraverso il bridge ad una porta ad alta velocità, ma questi 
pacchetti da ritrasmettere verso il cliente passano attraverso una porta di banda minore, quindi la porta più lente può andare facilmente in saturazione. 
Viene introdotto quindi tramite lo standard IEEE 802.3x e 802.3bd un controllo di flusso tramite dei ``pause frame'' un MAC control frame di 512 bit, per fermarsi 
prima di riprodurre traffico I pause frame non contengono dati ma contengono informazioni di controllo, e rappresentano una novità, viene quindi implementato 
attraverso un nuovo sottostato di MAC chiamato MAC control. Il supporto allo standard 802.3x è opzionale e viene negoziato tra le schede alle due estremità del filo. 

Prima dello standard 802.3x poiché ethernet era una comunicazione a turni, per impedire la saturazione i bridge potevano inviare pacchetti senza dati prendendo il 
controllo della connessione. 



\end{document}