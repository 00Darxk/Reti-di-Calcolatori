\documentclass{article}

\usepackage{cancel}
\usepackage{amsmath}
\usepackage[includehead,nomarginpar]{geometry}
\usepackage{graphicx}
\usepackage{amsfonts} 
\usepackage{verbatim}
\usepackage{mathrsfs}  
\usepackage{lmodern}
\usepackage{braket}
\usepackage{bookmark}
\usepackage{fancyhdr}
\usepackage{romanbarpagenumber}
%\usepackage{minted}
%\usepackage{subfig}
\usepackage[italian]{babel}
%\usepackage{float}
%\usepackage{wrapfig}
%\usepackage[export]{adjustbox}
\allowdisplaybreaks

\setlength{\headheight}{12.0pt}
\addtolength{\topmargin}{-12.0pt}
\graphicspath{ {./Immagini/} }

\hypersetup{
    colorlinks=true,
    linkcolor=black,
}

\newsavebox{\tempbox} %{\raisebox{\dimexpr.5\ht\tempbox-.5\height\relax}}


\makeatother

\numberwithin{equation}{subsection}
\newcommand{\tageq}{\tag{\stepcounter{equation}\theequation}}
\AtBeginDocument{%
  \renewcommand{\figurename}{Fig.}
}
\fancypagestyle{link}{\fancyhf{}\renewcommand{\headrulewidth}{0pt}\fancyfoot[C]{Sorgente del file LaTeX disponibile al seguente link: \url{https://github.com/00Darxk/Reti-di-Calcolatori/}}}

\begin{document}

\title{%
    \textbf{Reti di Calcolatori}  \\ 
    \large Appunti delle Lezioni di Reti di Calcolatori \\
    \textit{Anno Accademico: 2024/25}}
\author{\textit{Giacomo Sturm}}
\date{\textit{Dipartimento di Ingegneria Civile, Informatica e delle Tecnologie Aeronautiche \\
Università degli Studi ``Roma Tre"}}

\maketitle
\thispagestyle{link}

\clearpage


\pagestyle{fancy}
\fancyhead{}\fancyfoot{}
\fancyhead[C]{\textit{Reti di Calcolatori - Università degli Studi ``Roma Tre"}}
\fancyfoot[C]{\thepage}
\pagenumbering{Roman}

\tableofcontents

\clearpage
\pagenumbering{arabic}

\section{Introduzione}

Una qualsiasi interconnessione di calcolatori può rappresentare una rete di calcolatori, ma in base alla distanza reciproca tra questi componenti 
si tratta di reti differenti. Convenzionalmente si considerano reti di calcolatori, sistemi di calcolatori interconnessi ad una distanza 
superiore ai 50 cm. Una distanza minore, fino ai 5 cm, generalmente interessa componenti dello stesso computer, sulla stessa scheda madre, connesse tra di loro; mentre una 
distanza inferiore ai 5 cm rappresenta componenti sullo stesso chip. Inoltre le reti considerate possono essere ulteriormente divise in base 
alla distanza dei loro elementi:
\begin{itemize}
  \item Se hanno una distanza minore di 5 km, si tratta di risorse connesse sulla stessa rete o edificio, o su edifici vicini. Questo tipo di rete si chiama Local Area Network (LAN);
  \item Se hanno una distanza superiore ai 5 km, si tratta di risorse connesse su una vasta area geografica. Questo tipo di rete si chiama Wide Area Network (WAN).
\end{itemize}
Tra questi due livelli possono essere presenti anche tecnologie molto diverse tra di loro, queste tecnologie vengono identificate da acronimi da cui 
è possibile ricavare lo scopo della tecnologia, senza tuttavia conoscere il suo funzionamento. 

Una connessione tra componenti di una rete coinvolge sempre uno scambio di informazioni, tramite uno scambio di messaggi in serie. Gli elementi della 
rete effettuano degli accessi ad essa apparentemente in parallelo e simultanei, per poter comunicare tra di loro. Mentre su componenti sulla stessa macchina o sullo 
stesso chip avvengono tramite accessi ad una memoria condivisa. 

Le connessioni componenti di una rete avvengono su uno strato fisico, quindi attraverso diversi mezzi trasmissivi, i quali non verranno analizzati approfonditamente a 
questo livello di astrazione. Tra i più comuni mezzi trasmissivi abbiamo cavi in fibra ottica, o in rame, ed onde radio. 

\subsection{Commutazione}

All'interno di una rete si possono utilizzare due tipi diversi di commutazione, di circuito o di pacchetto. Il termine commutazione risale alla telefonia, quando 
diverse aree telefoniche dovevano essere collegate tra di loro, un operatore ad un centralino doveva manualmente collegare con un cavo le due aree interessate. 
Quest'operazione di connettere elementi tramite nodi intermedi in una rete rappresenta un elemento comune di tutte le reti. Collegare singolarmente tutti gli 
$n$ elementi di una rete comporterebbe un numero totale di $n(n-2)/2$ connessioni. All'aumento dei componenti in una rete il numero $n$ di connessioni da costruire sulla 
rete aumenta quadraticamente $O(n^2)$. Per cui è molto più conveniente mantenere nodi intermedi comuni a molti percorsi tra componenti della rete, utilizzando 
apparecchiature intermedie per realizzare circuiti. In questo modo però non è possibile soddisfare tutte le possibili coppie contemporaneamente, ma si elimina la 
crescita quadratica del sistema. 
La commutazione di circuito consiste nella creazione del percorso fisico che connette due elementi della rete. 







\end{document}