\documentclass{article}

\usepackage{cancel}
\usepackage{amsmath}
\usepackage[includehead,nomarginpar]{geometry}
\usepackage{graphicx}
\usepackage{amsfonts} 
\usepackage{verbatim}
\usepackage{mathrsfs}  
\usepackage{lmodern}
\usepackage{braket}
\usepackage{bookmark}
\usepackage{fancyhdr}
\usepackage{romanbarpagenumber}
%\usepackage{minted}
%\usepackage{subfig}
\usepackage[italian]{babel}
%\usepackage{float}
%\usepackage{wrapfig}
%\usepackage[export]{adjustbox}
\allowdisplaybreaks

\setlength{\headheight}{12.0pt}
\addtolength{\topmargin}{-12.0pt}
\graphicspath{ {./Immagini/} }

\hypersetup{
    colorlinks=true,
    linkcolor=black,
}

\newsavebox{\tempbox} %{\raisebox{\dimexpr.5\ht\tempbox-.5\height\relax}}


\makeatother

\numberwithin{equation}{subsection}
\newcommand{\tageq}{\tag{\stepcounter{equation}\theequation}}
\AtBeginDocument{%
  \renewcommand{\figurename}{Fig.}
}
\fancypagestyle{link}{\fancyhf{}\renewcommand{\headrulewidth}{0pt}\fancyfoot[C]{Sorgente del file LaTeX disponibile al seguente link: \url{https://github.com/00Darxk/Reti-di-Calcolatori/}}}

\begin{document}

\title{%
    \textbf{Reti di Calcolatori}  \\ 
    \large Appunti delle Lezioni di Reti di Calcolatori \\
    \textit{Anno Accademico: 2024/25}}
\author{\textit{Giacomo Sturm}}
\date{\textit{Dipartimento di Ingegneria Civile, Informatica e delle Tecnologie Aeronautiche \\
Università degli Studi ``Roma Tre"}}

\maketitle
\thispagestyle{link}

\clearpage


\pagestyle{fancy}
\fancyhead{}\fancyfoot{}
\fancyhead[C]{\textit{Reti di Calcolatori - Università degli Studi ``Roma Tre"}}
\fancyfoot[C]{\thepage}
\pagenumbering{Roman}

\tableofcontents

\clearpage
\pagenumbering{arabic}

\section{Introduzione}

Una qualsiasi interconnessione di calcolatori può rappresentare una rete di calcolatori, ma in base alla distanza reciproca tra questi componenti 
si tratta di reti differenti. Convenzionalmente si considerano reti di calcolatori, sistemi di calcolatori interconnessi ad una distanza 
superiore ai 50 cm. Una distanza minore, fino ai 5 cm, generalmente interessa componenti dello stesso computer, sulla stessa scheda madre, connesse tra di loro; mentre una 
distanza inferiore ai 5 cm rappresenta componenti sullo stesso chip. Inoltre le reti considerate possono essere ulteriormente divise in base 
alla distanza dei loro elementi:
\begin{itemize}
  \item Se hanno una distanza minore di 5 km, si tratta di risorse connesse sulla stessa rete o edificio, o su edifici vicini. Questo tipo di rete si chiama Local Area Network (LAN);
  \item Se hanno una distanza superiore ai 5 km, si tratta di risorse connesse su una vasta area geografica. Questo tipo di rete si chiama Wide Area Network (WAN).
\end{itemize}
Tra questi due livelli possono essere presenti anche tecnologie molto diverse tra di loro, queste tecnologie vengono identificate da acronimi da cui 
è possibile ricavare lo scopo della tecnologia, senza tuttavia conoscere il suo funzionamento. 

Una connessione tra componenti di una rete coinvolge sempre uno scambio di informazioni, tramite uno scambio di messaggi in serie. Gli elementi della 
rete effettuano degli accessi ad essa apparentemente in parallelo e simultanei, per poter comunicare tra di loro. Mentre su componenti sulla stessa macchina o sullo 
stesso chip avvengono tramite accessi ad una memoria condivisa. 

Le connessioni tra componenti di una rete avvengono su uno strato fisico, quindi attraverso diversi mezzi trasmissivi, i quali non verranno analizzati approfonditamente a 
questo livello di astrazione. Tra i più comuni mezzi trasmissivi abbiamo cavi in fibra ottica, o in rame, ed onde radio. 

\subsection{Commutazione}

Una rete di calcolatori può essere rappresentata come un grafo composto da vari nodi, per realizzare tutte le possibili coppie di calcolatori che potrebbero comunicare 
tra di loro attraverso la rete. Ma se venissero collegati individualmente tutte le possibili coppie di calcolatori necessiterebbe di infrastrutture massicce, poiché 
il numero dei possibili percorsi aumenta quadraticamente rispetto all'aumento dei calcolatori della rete. Infatti avendo $n$, tutte le possibili combinazioni tra 
questi calcolatori sono $n(n-2)/2$, nel caso ognuna di queste coppie corrisponda ad una connessione differente, il costo di costruzione e gestione della rete 
sarebbe eccessivo. 

Per risolvere questo problema e diminuire il numero totale di connessioni nella rete si utilizza il meccanismo della commutazione. Questo termine risale alla telefonia, 
dove ai sorse lo stesso problema, risolto introducendo centralini intermedi dove si potevano collegare diverse area telefoniche contenenti i telefoni che tentavano 
di comunicare. In questo modo si può drasticamente diminuire il numero di connessioni individuali nella rete, e non bisogna integrare un numero elevato di connessioni 
all'aggiunta di un singolo elemento. 
Si indica quindi con commutazione di circuito questo meccanismo di creare una connessione fisica tra due calcolatori, connettendo diverse zone 
della rete attraverso nodi intermedi. L'interazione tra computer consiste intrinsecamente da grandi quantità di dati trasmessi velocemente a grandi distanze, per cui 
hanno bisogno di infrastrutture dedicate massicce, si preferisce quindi questo sistema di nodi intermedi, nonostante non consenta di soddisfare contemporaneamente 
tutte le coppie di calcolatori. 

Poiché questa grande quantità di dati deve attraversare la rete velocemente, si utilizza una diversa tecnica di comunicazione a livello dei singoli messaggi, dividendoli 
in pacchetti da spedire separatamene. Nella commutazione a datagramma questi pacchetti vengono spediti su linee anche diverse e si mescolano a tutti i pacchetti che 
attraversano quel percorso. Le linee non sono quindi ad uso esclusivo di una singola connessione. 
Ma per ricomporre il messaggio originale bisogna combinare questi pacchetti nello stesso ordine in cui sono stati separati, sono necessari dati aggiuntivi per 
poter riconoscere il loro ordine, perso durante la trasmissione. La distanza attraversata da ciascun pacchetto infatti non è garantito sia uguale. 

Esiste inoltre un altro tipo di commutazione a circuito virtuale, dove i pacchetti vengono inviati sullo stesso percorso sequenzialmente, ed ogni linea può essere 
condivisa da un altro circuito virtuale, quindi non sono ad uso esclusivo. In questo caso invece è necessario un meccanismo per poter distinguere tra di loro 
questi circuiti virtuali sulla stessa linea fisica. 

Si è risolto tramite la commutazione di pacchetto l'esclusività delle linee della rete, introdotta dal modello a commutazione di circuito. 

%% TODO img commutazione di circuito, di pacchetto (datagramma e circuito virtuale)

La rete internet moderna utilizza la commutazione a datagramma, per motivi economici e gestionali. Altrimenti sarebbe necessario un gestore della rete che deve trovare 
un percorso ed attribuirlo ad una coppia ad ogni tentativo di connessione. Data la complessità della rete moderna, lasciare che i pacchetti trovino il percorso 
autonomamente è la scelta più efficiente. 
Per realizzare una commutazione a datagramma, piccole aree geografiche diverse vengono coperte da ``Internet Service Provider'' (ISP) differenti che possono comunicare 
solamente con altri ISP adiacenti. Quindi all'invio di un pacchetto, se il destinatario non è nella stessa zona dell'ISP corrente, questo lo invia ad un ISP adiacente 
che crede possa contenere il destinatario, così anche per la ricezione da un altro ISP. In caso il destinatario sia nella zona dell'ISP corrente, questo lo trasmette a lui. 
Accordi possono essere stipulati da chiunque a chiunque, un ISP ha sempre la necessità di trasmettere i pacchetti attraverso la rete. 

In certi casi l'ISP può gestire la rete a circuito virtuale, se sia il destinatario che il mittente siano coperti dal singolo ISP. 

\subsection{Velocità}

Nelle reti LAN e WAN  si possono trasmettere dati a velocità diverse:
\begin{itemize}
  \item LAN: velocità tra 10 ai 100 Mb/s;
  \item WAN: velocità tra 64 Kb/s ai 200-400 Mb/s. 
\end{itemize} 

Le reti WAN presentano hanno molti livelli di retrocompatibilità mantenuti, per cui si possono trasmettere dati a velocità minori di una rete LAN. 
In generale sono sempre richieste reti a velocità di trasmissione elevata, e connessioni ad alta velocità. 

Data una rete si può definire la velocità in due modi differenti. Se si considera il tempo in cui il primo bit del messaggio arriva a destinazione. Le connessioni ad 
alta velocità vengono realizzate in linee a fibra ottica, per cui i bit vengono inviati come impulsi di luce, e viaggiano ad una velocità costante, quindi per ogni rete 
la velocità di trasmissione di un singolo bit è la stessa. Si definisce quindi il tempo di ritardo o delay, il tempo per trasmettere un singolo bit, alla velocità della luce, sulla 
rete e dipende interamente dalla distanza. 

Un pacchetto non viene rappresentato da un singolo bit, per cui non possono essere trasmessi alla stessa velocità, si definisce banda la quantità di bit trasmessi 
contemporaneamente sulla linea. Questa si chiama banda, e generalmente è sempre possibile comprare più banda in modo relativamente facile, ma è molto difficile comprare meno delay. 

\subsection{Gestione delle Risorse}

La rete è essenzialmente un insieme di risorse interconnesse tra di loro e dalla teoria dei sistemi operativi, il loro controllo può essere descritto da varie attività:
\begin{itemize}
  \item Verifica dei diritti d'accesso;
  \item Sequenziamento degli accessi alla risorsa;
  \item Esecuzione delle operazioni disponibili. 
\end{itemize}

Ad ogni risorsa vengono assegnato almeno un gestore, di numero variabile in base al tipo di gestione. 
Le modalità di gestione delle risorse sono varie, si dividono in gestione autocratica e multilaterale. Nella gestione autocratica ogni risorsa ha un unico gestore 
associato ed univoco. Nella gestione multilaterale per ogni risorsa può esserci più di un gestore, si possono identificare quindi tre sottotipi di questa gestione:
\begin{itemize}
  \item Gestione partizionata, dove attività di gestione viene effettuata da un singolo processo;
  \item Gestione successiva, dove tutte le attività di gestione vengono effettuate a turni da più processi;
  \item Gestione replicata, dove tutti i gestori partecipano a ciascuna attività, se ogni gestore ha peso decisionale uguale allora si tratta di gestione democratica. 
\end{itemize}

La gestione replicata fornisce una forte resistenza ai guasti per un numero elevato di gestori che partecipano a ciascuna istanza di una attività, con alto grado di 
uguaglianza nella responsabilità di gestione. Sono abbastanza diffusi meccanismi di elezione per la scelta dei gestori. 

\clearpage

\section{Modello ISO-OSI}

Per il funzionamento della rete gli standard sono strettamente necessari, altrimenti non sarebbe possibile una comunicazione tra un mittente ed un destinatario qualsiasi, 
alcuni di questi standard vengono imposti dalla case costruttrici, altri vengono definiti da organizzazioni internazionali, nell'ambito informatico o delle 
telecomunicazioni. Alcune di queste associazioni come IETF sono indipendenti da stati nazionali, dove varie aziende o istituzioni propongono modifiche di vecchi 
standard o introduzione e definizione di nuovi. 

Il modello ISO-OSI rappresenta un importante strumento di classificazione nel modo delle reti. Venne realizzato in parte e sostanzialmente dismesso, ma nonostante 
questo viene utilizzato a livello globale. 

Questo modello si basa sull'architettura stratificata di hardware o software, dove partendo da un nucleo centrale il sistema viene diviso in livelli o strati 
indipendenti dal livello inferiore, ed uno strato fornisce servizi solamente allo strato immediatamente superiore. Avanzando da uno strato al superiore i servizi vengono mostrati in modo 
sempre più astratto ed il sistema aumenta progressivamente di utilità. Per la sua utilità questo tipo di architettura stratificata permane molti campi dell'informatica. 

La rete viene divisa in 7 livelli numerati dal basso verso l'alto, il livello indica la funzione delle tecnologie che vi appartengono e questo fornisce uno strumento di 
classificazione per analizzarle senza dover conoscere i loro meccanismi interni. 

%% TODO img tabella strati iso-osi

Ogni strato rappresenta un diverso livello di astrazione ed offrono funzioni ben definite. Poiché ognuno di questi strati è indipendente dal livello inferiore, viene minimizzato lo scambio 
di informazioni tra strati. Il numero dei livelli venne scelto in base alle funzioni distinte di una rete da descrivere e dalla realizzabilità. 

All'interno di ogni strato si possono individuare diverse ``entità'', hardware o software dove sono contenuti i protocolli di quel livello. Per offrire servizi allo 
strato superiore, è presente un punto logico chiamato ``Service Access Point'' (SAP) al quale può accedere il livello superiore. 
L'unico punto di contatto tra livelli e quello inferiore è la loro interfaccia. Un protocollo è un linguaggio utilizzato da entità dello stesso livello, quindi entità 
di uno stesso strato possono comunicare con le adiacenti tramite protocolli e con superiori tramite SAP, ed inferiori tramite interfaccia. 

Secondo questo modello i pacchetti sono contenuti in altri pacchetti, destinati a livelli inferiori, per cui quando vengono ricevuti da un livello $n-1$, viene letto 
il pacchetto di livello $n-1$ ed estratto il pacchetto di livello $n$ contenuto ed inviato all'entità di livello $n$. 
I protocolli su uno stesso strato possono comunicare con altre entità dello stesso livello, e possono inviare indicazioni o conferme a richieste di entità utenti del 
servizio del livello superiore. 

I dati generati da un protocollo di livello $n$ sono detti $n$-pdu, ``Protocol Data Unit'', composti da un header indirizzato all'entità di livello $n$ ed una payload, 
contente un $n+1$-pdu, destinata al livello superiore. Per cui all'aumento dei livelli aumenta l'overhead. 

I diversi livelli di questo modello presentano le seguenti funzioni:
\begin{enumerate}
  \item  Il primo strato della pila ISO-OSI rappresenta lo strato fisico, che si interfaccia direttamente con il mezzo trasmissivo della rete e quindi rappresenta il livello di 
  natura fisica della trasmissione. Offre al livello superiore una comunicazione indipendente dal mezzo trasmissivo. Fornisce allo strato di collegamento servizi di 
  trasmissione di bit a tra sistemi adiacenti, consegna in sequenza di bit o notifiche di malfunzionamenti. 
  \item Il secondo livello rappresenta lo strato data-link per risolvere eventuali malfunzionamenti dello strato fisico, rilevando e correggendo errori, tramite algoritmi di correzione, come i bit di parità. 
  Offre allo strato superiore la possibilità di trasmettere pdu a sistemi adiacenti utilizzando due code nelle due direzioni. 
  \item Il terzo livello è lo strato di rete e conosce la topologia completa della rete, per effettuare operazioni di instradamento. Contiene i protocolli come IPv4, 
  progressivamente sostituito da IPv6, e permette il trasferimento di pdu da estremo ad estremo. Inoltre permette la commutazione di circuito o di pacchetto a datagramma e 
  a circuito virtuale. 
  \item Il quarto livello di trasporto, divide il messaggio in pacchetti, prova a colmare fluttuazioni della qualità del servizio dello strato di rete in modo 
  trasparente rispetto agli strati superiori. In caso manchino dei pacchetti prova a recuperarli attraverso algoritmi di correzione, è il primo strato che risiede 
  solamente nei terminali. Offre allo strato superiore la possibilità di instaurare una connessione e gestione della stessa, una trasmissione affidabile, ed il rilascio 
  della connessione. 
  \item Il quinto livello di sessione sincronizza e struttura il dialogo tra due processi.
  \item Il sesto livello di presentazione permette uno scambio di messaggi indipendentemente dalla sintassi della trasmissione. 
  \item Il settimo livello di applicazione offre un mezzo per accedere alla rete tramite un processo, interfacciando l'utente alla rete.  
\end{enumerate}

Per tutti i livelli superiori a quello fisico si possono definire due modalità operative ed associati servizi e protocolli, connessi e non connessi. Nei servizi o 
protocolli connessi, si instaura una connessione o dialogo tra le entità, e termina solamente dopo convenevoli finali. La modalità non connessa non ha bisogno di una 
connessione costante tra le due entità, viene instaurata senza un dialogo da una delle entità senza una terminazione. 

Nella prima modalità l'entità non ha bisogno di ascoltare tutto il traffico per determinare quali pdu sono indirizzati alla stessa, ma necessita di una connessione 
continua anche se vengono trasmessi una piccola quantità di dati. Mentre nella seconda modalità possono essere inviate pdu indipendente dalla connessione e dalla 
distanza temporale tra le due, ma le entità che offrono questo servizio o protocollo devono costantemente analizzare il traffico per individuare le pdu a loro 
indirizzate. 
I protocolli non connessi sono quindi più efficienti, ma mancano di affidabilità, poiché manca una conferma di ricezione dei dati come nei protocolli connessi. quest'ultimi 
sono quindi più affidabili, ma meno efficienti, poiché dopo aver instaurato il dialogo non è possibile terminarlo preventivamente, e ciò può causare uno spreco di risorse. 


In un servizio connesso sono presenti primitive per instaurare una connessione, inviare messaggi e una conferma di ricezione o ricevere messaggi, specificare 
l'indirizzo o nome della connessione ed abbattere la connessione. Mentre per servizi non connessi sono presenti solo primitive per inviare messaggi separatamene. 
Nelle reti LAN sono disponibili servizi connessi, solamente sul quarto strato, mentre nelle reti WAN è possibile siano offerti anche nel primo strato.  



I primi tre livelli della pila ISO-OSI sono presenti su ogni nodo della rete non solo sui calcolatori, poiché rappresentano i livelli di trasmissione dei pacchetti, 
necessari anche nei nodi intermedi per poter trasmettere i pacchetti. 

Nei protocolli di livello 2,3 e 4 si utilizzano meccanismi di riscontro o acknowledgment e tecniche di controllo a finestra, a riga indice e puntatore in avanti per 
correggere eventuali errori nei pacchetti. 

Gli ultimi tre strati della pila si interfacciano con le applicazioni e lavorano generalmente in parallelo invece che in serie come il resto della pila. 

Una singola connessione di livello $n$ può essere sfruttata da più connessioni di livello $n+1$, interne in modo che gli $n$-pdu contengono entrambe le $n+1$-pdu delle due 
connessioni. Può essere il caso di connessioni tra più processi diversi sulle stesse due macchine, dove una singola connessione tra queste due macchine contenga numerose 
connessioni processo-processo tra le due. 

Inoltre una singola connessione $n+1$ può utilizzare più di una connessione di livello $n$, per parallelizzare la trasmissione e velocizzarla partizionando i dati da 
inviare, oppure per implementare una resistenza ai guasti. La connessione $n+1$ utilizza più canali di comunicazione di livello $n$ non in competizione. 

Una singola connessione di livello $n+1$ nel tempo, può utilizzare più di una connessione di livello inferiore; è possibile che il terminale si sposta durante la 
trasmissione e si aggancia a reti diverse da quella iniziale, senza interrompere la connessione. 
Nello stesso caso, una stessa connessione di livello $n$ continua nel tempo può essere utilizzata da diverse connessioni di livello $n+1$. La connessione originale dell'esempio 
precedente vede uscire il primo terminale e quindi la prima connessione $n+1$ per poi vedere accedere un altro terminale ed un'altra connessione $n+1$. 

\clearpage

\section{Standard IEEE 802}

Lo standard IEEE 802 riguarda i primi due livelli del modello ISO-OSI, ovvero il livello fisico, ed il livello data link. Le tecnologie definite in base a questo standard quindi hanno come obiettivo la trasmissione 
e la rivelazione o correzione di bit attraverso un mezzo trasmissivo. Si occupano della connessione e quindi comunicazione tra macchine adiacenti, per una qualche definizione di adiacenza. Altri protocolli e standard 
noti sono l'IPv4 ed IPv6, protocolli di routing di livello tre, protocolli TCP ed UDP di livello quattro, ed il protocollo HTTP di livello sette, ma si occupa anche da solo delle funzioni dei livelli 5 e 6. 

Questi standard vengono realizzati dall'organizzazione IEEE, Institute of Electrical and Electronics Engineers, organizzazione indipendente da stati sovrani. Il progetto IEEE 802 venne definito con 
l'obiettivo di realizzare una serie di standard di livello fisico e data-link per permettere la comunicazione di calcolatori sulla stessa rete locale, LAN, personale, PAN, o rete metropolitana, MAN, di grandezza 
intermedia tra le reti LAN e WAN. Ha avuto successo sopratutto per le reti LAN e MAN, ma per le reti personali si utilizza uno standard diverso basato sul bluetooth. 
Questi standard riguardano tecnologie con pacchetti di lunghezza variabile.


Le specifiche tecnologie vengono individuate tramite una notazione puntata, con 802.$x$, dove $x$ rappresenta un numero, ed identifica la tecnologia. I numeri precedenti al punto individuano lo standard dove è 
stata introdotta questa tecnologia. Ma le tecnologie non rimangono invariate nel tempo, per cui si possono assegnare delle lettere dopo il numero per specificare la versione o tipo di quella specifica tecnologia. 

Lo standard IEEE 802 divide il livello due in due sottolivelli: ``Logical Link Control'' (LLC) e ``Media Access Control'' (MAC), questi gestiscono due tipologie diverse di pacchetti. Per le diverse tecnologie dello standard, il livello MAC è diverso, mentre il livello 
LLC è comune a tutti. 
Il sottolivello MAC è specifico per ogni tipo di LAN, si suppone che tutti i calcolatori che devono comunicare siano nella stessa LAN. Data questa ipotesi il sottolivello MAC risolve il problema di determinare 
il destinatario in ricezione, e di verificare la disponibilità della LAN in trasmissione, in caso la LAN sia a singolo canale condiviso. Quindi bisogna evitare che il canale sia utilizzato da più utenti. 

\subsection{Sottolivello MAC}

Poiché il canale è condiviso, tutti gli utenti possono vedere i pacchetti inviati, sono quindi necessari protocolli di sicurezza e cifratura per impedira che sia possibile a chiunque connesso alla rete leggere il 
contenuto dei pacchetti. Tecniche che non verranno trattate in questo corso. 
Inoltre si utilizza un canale condiviso poiché se ci fosse un malfunzionamento fisico, una sola connessione verrebbe compromessa e non l'intera rete. 

Per determinare il destinatario di un pacchetto nella MAC pdu è presente un campo per definire il tipo di trasmissione:
\begin{itemize}
  \item Punto a Punto: da un calcolatore ad un altro nella LAN;
  \item Punto a Gruppo: da un calcolatore a diversi altri nella LAN;
  \item Broadcast: a tutti gli utenti connessi alla LAN. 
\end{itemize}

Per permettere di identificare univocamente un unico elemento nella rete, gli indirizzi MAC devono essere univoci nella rete considerata. Dato che è possibile connettersi ad una LAN dall'esterno senza conoscere gli 
indirizzi MAC utilizzati, servirebbe un gestore di rete per assegnarli ad ogni nuova connessione, ma questo è un approccio inefficiente. Si utilizzano quindi indirizzi MAC univoci a livello mondiale, in questo modo 
nell'intera rete esisteranno solo indirizzi MAC differenti. Questa condizione vale anche su VPN o LPN, inoltre se su una stessa macchina vengono simulate diverse macchine virtuali, ognuna di esse dovrà avere un 
indirizzo MAC differente, all'interno della rete locale che utilizzano per comunicare tra di loro. 
Se due macchine avessero lo stesso MAC, riceverebbero gli stessi pacchetti, e verrebbero riconosciuti da entrambe le macchine come propri. 



La MAC pdu è composta da diversi campi, che possono variare in base alla tecnologia con l'aggiunta di campi specifici. Ma per ogni tecnologia aderente allo standard IEEE 802 sono presenti sicuramente questi quattro 
campi per la MAC pdu:
\begin{itemize}
  \item MAC-dsap (Destination Service Access Point): indirizzo di destinazione;
  \item MAC-ssap (Source/Send Service Access Point): indirizzo di partenza;
  \item Info: LLC pdu;
  \item FCS (Frame Check Sequence): per identificare e correggere eventuali errori.
\end{itemize}

%% TODO img struttura mac pdu (divisione campi)

Per ottenere l'indirizzo del destinatario, si utilizzano protocolli di acquisizione descritti in seguito. 

Gli indirizzi MAC sono composti da 6 byte, in base allo standard EUI-48, ``Extended Unique Identifier'', per indirizzi a 48 bit, ma esiste anche uno standard a 64 bit non utilizzato. Questi byte vengono rappresentati in forma esadecimale, 
separati da due punti o trattini. I primi tre byte dell'indirizzo MAC vengono assegnati al costruttore e rappresentano gli OUI ``Organization Unique Identifier'', gli ultimi 3 byte vengono scelti dal costruttore. 
Per cui dato un indirizzo MAC, è sempre possibile determinare il costruttore della macchina a cui appartiene. 

Esistono diversi tipi di indirizzi MAC, in base al valore di determinati bit:
\begin{itemize}
  \item Unicast: indirizzi che individuano le singole schede di rete dei calcolatori; se l'ultimo bit del primo byte ha valore zero;
  \item Multicast: indirizzi che identificano gruppi di schede di rete; se l'ultimo bit del primo byte ha valore uno;
  \item Broadcast: identificano tutte le schede di rete; se l'indirizzo è \verb|FF:FF:FF:FF:FF:FF|.  
\end{itemize}

Inoltre è possibile assegnare indirizzi MAC non unici a livello mondiale, specificando il valore del penultimo bit del primo byte, in modo che abbia valore uno. In questo modo è possibile gestire localmente indirizzi 
MAC nella stessa LAN. 

Per risolvere i conflitti in trasmissione si utilizzano nelle rete WiFi degli algoritmi distribuiti sulle singole macchine in contemporanea, che collaborano per determinare a chi abilitare gli accessi alla rete. 

\subsection{Sottolivello LLC}

Il sottolivello LLC consegna al livello MAC un pacchetto da spedire, questo pacchetto è uguale per ogni tecnologia aderente allo standard e presenta campi analoghi al sottolivello MAC. I campi LLC-dsap/ssap individuano 
gli indirizzi LLC del mittente e destinatario, il campo info contiene il tipo di pdu per gestire diverse tipologie di pacchetto, e l'ultimo campo contiene la pdu di terzo livello. 
Gli indirizzi LLC non individuano macchine come gli indirizzi MAC, ma vengono utilizzati per identificare i protocolli di livello 3 a cui sono indirizzati i pacchetti. Consentono la convivenza di diversi protocolli 
di livello 3 sulla stessa macchina e sulla stessa LAN, dove sono presenti diverse pile protocollari con obiettivi e versionatura diversa. 

Gli indirizzi LLC vengono attribuiti dall'IEEE solo a protocolli ``ufficialmente'' standard, ma questa è una classificazione che ignora molti dei protocolli più utilizzati a livello globale come TPC-IP, il 
protocollo più utilizzato al mondo. Vengono identificati da un byte in esadecimale, se non sono protocolli standard, il loro indirizzo è \verb|AA|, ed il pacchetto subisce una variazione con la snap-pdu, 
``SubNet Access Point'', dopo il campo control di 5 byte per identificare il protocollo. Questo rappresenta un ulteriore livello di overhead, già elevato per il modello ISO-OSI, per i pacchetti. 

%% TODO img llc pdu dentro a mac pdu

\subsection{802.3: Ethernet}

%% TODO aggiungere lezioni 4, 9 ott


La tecnologia ethernet ha avuto un enorme successo come strumento di comunicazione di reti LAN, e WAN. 

%% add

Un pacchetto ethernet deve comunque avere 512 bit da trasmettere, altrimenti la parte residua del campo dati deve essere riempita di bit 
di riempimento chiamato padding, inoltre il campo dati può contenere al massimo 1500 Byte. 

\subsection{802.1D: Bridge-Switch}

La connessione instaurata tramite ethernet è bidirezionale simultanea solamente su due calcolatori, ma su una stessa connessione può inviare i dati un solo 
calcolatore, quindi sono necessarie altre componenti. Un bridge è una componente che consente di connettere tra di loro più di un computer 
tramite ethernet, comportandosi come se fosse un calcolatore intermedio ai calcolatori della rete, connesso a ciascuno di questi tramite una connessione 
ethernet. 

Inoltre connettendo tra di loro diversi bridge è possibile creare una struttura più articolata, creando una struttura simile ad un albero. La 
parte wired o cablata delle connessioni LAN vengono instaurate in questo modo. 

%% TODO img rete bridge


I bridge svolgono una prima funzione di rendere possibili topologie articolate, effettuando un'operazione di ``filtering'', per separare tra di loro porzioni di rete che 
non devono dialogare tra di loro in modo diretto

%% TODO img filtering

I bridge sono delle macchine ``store \& forward'', ovvero quando ricevono un pacchetto, prima di essere inviato su altre porte, viene 
memorizzato e trasmesso su altre porte, analogamente come se fosse un calcolatore, in caso le altre porte siano impegnate a trasmettere altri 
pacchetti, quindi in caso di traffico. Si può quindi immaginare una coda di pacchetti sulle porte del bridge per essere trasmesse. 

Il bridge sono delle tecnologie di livello 2, ed utilizzano algoritmi di instradamento per inviarli ad un MAC address specifico, ma questo tipo di 
algoritmo viene effettuato a livello 3. Questo non sorge problemi, poiché quest'operazione di instradamento è interna alla LAN, e non coinvolge alcun'altra componente 
della rete. I bridge devono essere conformi allo standard IEEE 802.1D. Gli standard comprendenti il carattere ``D'', sono di grande importanza. 
I sistemi connessi a reti LAN ignorano i bridge, si dicono quindi trasparenti, poiché i calcolatori connessi alla rete non conoscono la loro posizione all'interno della 
rete. 

%% TODO img connessione lan dallo standard 

Un calcolatore per inviare un messaggio ad un altro calcolatore su una rete LAN, invia il suo pacchetto ad un bridge attraverso il sul MAC address. Il bridge quindi 
utilizza in principio un diverso MAC address, per spedire questo pacchetto al computer di destinazione tramite il suo MAC address. Tra questi due MAC è presenta un 
componente di relay, per trasmettere il pacchetto tra porte diverse del bridge. 


Le porte di un bridge possono avere lo stesso MAC o MAC differenti. Poiché il pacchetto è specifico al MAC del bridge, deve ricostruire il pacchetto scartando i campi 
specifici al MAC address del bridge. Inoltre poiché i pacchetti non sono tutti conformi allo standard IEEE 802.3, deve ricostruire anche il campo LLC. I MAC address 
dei computer nella rete sono realizzati in modo da poter essere connessi a ciascun tipo di MAC. 

%%\subsubsection{Learning}

Si vuole che il modello di rete sia ``plug \& play'', ovvero non deve essere dipendente da un intervento umano. 
I bridge costruiscono la loro tabella di instradamento per identificare dove sono presenti i diversi MAC address, autonomamente attraverso un meccanismo di ``learning'', 
salvando questa tabella nel ``filtering database''. Ogni porta del bridge rappresenta una linea ethernet diversa, identificando un loro dominio di collisione, a cui 
possono essere connessi diversi calcolatori. 

Si considera una rete dove ogni componente connesso è spento, ed una tabella vuota. Appena si accende un calcolatore ed invia un pacchetto da un dominio di collisione, 
allora il bridge capisce a quale porta corrisponde il MAC address del mittente. Ma ancora non conosce dove si trova il destinatario, quindi lo invia su tutte le sue 
porte disponibili, su tutta la rete. Invece se conosce la porta dov'è presente il destinatario lo invia solamente su quella porta. 

%% TODO img tabella di filtering

Il learning permette di costruire autonomamente il filtering database di un bridge, questo meccanismo tuttavia non funziona quando al rete presenta una topologia diversa 
dalla topologia ad albero. Per esempio se è presente un ciclo all'interno della rete, il bridge si vede arrivare un pacchetto dallo stesso MAC address su porte diverse. 
Ma un albero è una topologia contenente solo ``Single Points of Failures'' (SPoF) e quindi fortemente sconsigliata, poiché un singolo malfunzionamento causerebbe la 
perdita di funzionalità dell'intera rete. Per cui data una topologia a grafo, un bridge è in grado di calcolare autonomamente un albero 
ricoprente della rete, ad ogni cambio di topologia della stessa. I bridge inoltre vengono collegati tra di loro per più di una connessione per evitare altri SPoF, ed 
evitare che a un singolo guasto la rete venga tagliata in due. 

Tramite un meccanismo progressivo i bridge individuano la loro posizione nella struttura dell'albero ricoprente e sono in grado di staccare alcune porte e rimanere 
collegati sull'intera rete; quando questi bridge rilevano un guasto su una di queste connessioni, riattivano una delle porte disattivate per mantenere in funzione la 
rete, ottenendo una significativa resistenza ai gusti. 
Questo tipo di algoritmo di spanning tree verrà trattato in corsi più avanzati di reti di calcolatori. 

Le prestazioni di un bridge influenzano le prestazioni dell'intera rete locale, vengono identificate da una serie di parametri. Il numero massimo di pacchetti al 
secondo processabili dal bridge, rappresentano un collo di bottiglia per i pacchetti che possono essere presenti sulla rete in ogni singolo momento, se vengono 
inviati un numero superiore ci pacchetti, alcuni pacchetti verranno scartati. Un altro parametro caratteristico è il tempo medio di latenza, ovvero il tempo in cui il 
bridge prende le sue decisioni ed invia il pacchetto alla porta giusta. 
Per cui è preferibile avere bridge full speed, ovvero con una velocità pari al massimo teorico. Più corti sono i pacchetti maggiore è il numero di decisioni 
effettuate nell'unità di tempo. Nello standard IEEE 802.3 a 10 Mb/s, un bridge si definisce full speed, se è in grado di processare 41880 pacchetti al secondo, 
per ogni porta. Questi esperimenti di verifica a parità di frequenza devono essere effettuati utilizzando pacchetti di lunghezza minima, così ad ogni porta è presente 
la massima frequenza di funzionamento del bridge. 
Il pacchetto più piccolo che può essere inviato è da 512 bit, per cui il numero massimo di pacchetti al secondo ad una 
velocità di 10 Mb/s è di circa 19500 pacchetti, ma il pacchetto comprende anche il preambolo ed lo SFD, per cui vanno aggiunti altri 64 bit, 
numero di pacchetti al secondo scende quindi a 17300. 
Tuttavia tra un pacchetto ed il successivo in ethernet è presente l'``interpacket gap'' di 96 bit. 
Per ciascuna tipologia di porta del bridge si effettua questa analisi e si verifica nel caso peggiore quanti pacchetti è in grado di gestire. 


Il bridge è un calcolatore, con una CPU, RAM ed interfacce per le diverse LAN, in ROM le funzionalità dello standard %%
Per bridge più potenti, le  porte vengono realizzate tramite schede ASIC, per risolvere il problema dell'instradamento localmente. Le porte vengono realizzati tramite 
diversi slot che possono essere inseriti o rimossi in base al tipo di porta necessaria. Inoltre sono necessari massicci impianti di raffreddamento per riuscire a 
mantenere la temperatura del data center. A differenza di un server che può essere rallentano in caso di traffico allentato e quindi diminuire la temperature, le apparecchiature di bridge non possono 
spegnersi, e quindi comportano una temperatura costantemente elevata. Bridge di fascia alta sono in grado di effettuare bilanci sulle prese di corrente. 

Logicamente sono presenti almeno due porte una ``MAC relay entity'', per trasmettere i pacchetti tra le varie porte, ed un'entità di livello superiore, per la gestione 
del bridge, degli algoritmi e dei protocolli. Queste entità di alto livello comunicano con altri bridge attraverso pacchetti per realizzare lo spanning tree. 

Le porte del bridge possono essere abilitate o disattivate dall'amministratore di rete. Una porta attiva può essere in stato di ``forwarding'' o di ``blocking'', se sono 
bloccate lo spanning tree lo ha bloccate. Ogni porta ha un indirizzo MAC univoco, e sono numerate progressivamente a partire da uno. Convenzionalmente l'indirizzo MAc 
del bridge corrisponde all'indirizzo MAC della porta numero uno. 

La tabella di instradamento contiene ``entries'' (righe) statiche o dinamiche. Le righe statiche vengono inserite dall'amministratore a causa di esigenze di sicurezza 
importanti, altrimenti la posizione del MAC address vengono mantenuti per un tempo finito, configurabile, e di default di 5 minuti. Infatti è possibile che il calcolatore 
venga spostato spazialmente attraverso la rete, è quindi possibile si colleghi ad una porta differente. 

%% add ??

Lo sviluppo di ethernet ha portato alla creazione di meccanismi di controllo di flusso, soprattutto per gli switch. Una richiesta attraverso il bridge può essere di 
pochi byte verso un server, ma può provocare un trasferimento notevole di dati verso il client, quindi attraverso il bridge ad una porta ad alta velocità, ma questi 
pacchetti da ritrasmettere verso il cliente passano attraverso una porta di banda minore, quindi la porta più lente può andare facilmente in saturazione. 
Viene introdotto quindi tramite lo standard IEEE 802.3x e 802.3bd un controllo di flusso tramite dei ``pause frame'' un MAC control frame di 512 bit, per fermarsi 
prima di riprodurre traffico I pause frame non contengono dati ma contengono informazioni di controllo, e rappresentano una novità, viene quindi implementato 
attraverso un nuovo sottostato di MAC chiamato MAC control. Il supporto allo standard 802.3x è opzionale e viene negoziato tra le schede alle due estremità del filo. 

Prima dello standard 802.3x poiché ethernet era una comunicazione a turni, per impedire la saturazione i bridge potevano inviare pacchetti senza dati prendendo il 
controllo della connessione. 

\subsection{802.11: WiFi}

%% add


\end{document}